\documentclass[11pt]{report}

\usepackage{graphicx}
\usepackage{amsmath}
\usepackage{hyperref}
\usepackage{geometry}
\usepackage{listings}
\usepackage{xcolor}
\geometry{a4paper, margin=0.3in}


\begin{document}
\begin{lstlisting}[language=C++, caption=Original code listing]
int main(int argc, char* argv[]) {
    int nx(10000), ny(200), nt(200);        
    double** vi=new double*[nx];
    double** vr=new double*[nx];
    double pi=(4.*atan(1.));        Constant here? Could use M_PI 
                                    Direct initialization?

    for(int i=0;i<nx;i++) {         Should probably change the operation order 
        vi[i]=new double[ny];       here with the self incrementation ie. ++i
        vr[i]=new double[ny];       instead of ++i. 
    }
                                    Cache thrashing? Try offsetting
    for(int i=0;i<nx;i++) {         the memory access pattern 
        for(int j=0;j<ny;j++) {       
            vi[i][j]=double(i*i)*double(j)*sin(pi/double(nx)*double(i));  pi/double(nx) x
            vr[i][j]=0.;                                                  in the loop is inefficient
        }
    }

    ofstream fout("data_out");

    for(int t=0;t<nt;t++) {
        cout<<"\n"<<t;cout.flush();
        for(int i=0;i<nx;i++) {
            for(int j=0;j<ny;j++) {
                if(i>0&&i<nx-1&&j>0&&j<ny-1) {
                    vr[i][j]=(vi[i+1][j]+vi[i-1][j]+vi[i][j-1]+vi[i][j+1])/4.;
                } else if(i==0&&i<nx-1&&j>0&&j<ny-1) {
                    vr[i][j]=(vi[i+1][j]+10.+vi[i][j-1]+vi[i][j+1])/4.;
                } else if(i>0&&i==nx-1&&j>0&&j<ny-1) {
                    vr[i][j]=(5.+vi[i-1][j]+vi[i][j-1]+vi[i][j+1])/4.;
                } else if(i>0&&i<nx-1&&j==0&&j<ny-1) {
                    vr[i][j]=(vi[i+1][j]+vi[i-1][j]+15.45+vi[i][j+1])/4.;
                } else if(i>0&&i<nx-1&&j>0&&j==ny-1) {
                    vr[i][j]=(vi[i+1][j]+vi[i-1][j]+vi[i][j-1]-6.7)/4.;
                }
            }
        }

        for(int i=0;i<nx;i++) {
            for(int j=0;j<ny;j++) {
                if(fabs(fabs(vr[i][j])-fabs(vi[i][j]))<1e-2) fout<<"\n"
                <<t<<" "<<i<<" "<<j<<" "
                <<fabs(vi[i][j])<<" "<<fabs(vr[i][j]);
            }
        }

        for(int i=0;i<nx;i++) {
            for(int j=0;j<ny;j++) vi[i][j]=vi[i][j]/2.+vr[i][j]/2.;
        }
    }
}
\end{lstlisting}

My initial changes consisted of:
\begin{itemize}
    \item Changing the operation order in the for loop to use the pre-increment operator instead of the post-increment operator.
    \item Initialized pi as a constant double.
    \item Ln 28: Removed unnecessary recalculation of $pi / double(nx)$.
\end{itemize}
First run after these changes shaved almost 3 full seconds from the initial run-time however this doesn't seem to be a consistent result.

\end{document}